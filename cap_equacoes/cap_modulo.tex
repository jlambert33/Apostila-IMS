 \chapter{Módulo}

 \section{Expressões algébricas modulares}

 Dois números reais podem ser associados por um número real chamado de ``distância''. Podemos observar (de modo intuitivo) que a distância dos pontos $-x$ e $x$ até a origem (zero) é a mesma e, matematicamente, chamada de \textbf{valor absoluto}\index{Valor absoluto|see{Módulo}}\index{Módulo}, ou \textbf{módulo}, do número $x$, e é representada por $\abs{x}$. Assim, dizemos que:
\begin{itemize}
\item O valor absoluto de $-3$ é $3$, ou seja, $\abs{-3}= 3$ (a distância do -3 até a origem é 3);
\item O valor absoluto de $3$ é $3$, ou seja, $\abs{3}= 3$ (a distância do 3 até a origem é 3);
\end{itemize}
\begin{obs}
Generalizando esta ideia definimos que:
\begin{equation}
\label{eq:modulo}
\abs{x}= \left\{\begin{array}{rl}
      -x , & \text{se} \ \ x<0 \\
      x , & \text{se} \ \ x \geq 0
     \end{array}\right. .
     \end{equation}
\end{obs}

Em termos práticos, quando analisamos $|x|$ pelo lado direito da Equação \ref{eq:modulo}, dizemos que ``abrimos o módulo''.

Se precisamos trabalhar com o valor absoluto ou módulo de um número real, levamos em conta as seguintes propriedades:

\begin{prop}[Propriedades do módulo]
 Para quaisquer $x, y \in \R$, são válidas as seguintes propriedades: \label{prop.modulo}
\begin{enumerate} \begin{multicols}{2}
 \item $\abs{x} \geq 0$;
 \item $\abs{x}= 0 \Leftrightarrow x= 0$;
 \item $x \leq \abs{x}$;
 \item $-x \leq \abs{x}$;
 \item $\abs{-x}= \abs{x}$;
 \item $\abs{x}^2= x^2$;
 %, de fato,\\
 %se $x \geq 0$, pela definição do módulo temos que $\abs{x}= x$ e daí $\abs{x}^2= x^2$, \\
 %se $x < 0$, pela definição do módulo temos que $\abs{x}= -x$ e daí $\abs{x}^2=(-x)^2= x^2$.\\
 %Portanto, para todo $x \in \R$, $\abs{x}^2= x^2$.

 \item $\abs{x^n}= \abs{x}^n$, se $n$ é par;%, de fato, \\
 %se $x \geq 0$, pela definição do módulo temos que $\abs{x}= x$ e daí $\abs{x}^n= x^n$, \\
 %se $x < 0$, pela definição do módulo temos que $\abs{x}= -x$ e daí $\abs{x}^n= (-x)^n= x^n$.\\
 %Portanto, para todo $x \in \R$, $\abs{x}^n= x^n$.

 \item $\abs{x \cdot y}= \abs{x} \cdot \abs{y}$;%, de fato,
%\begin{equation*}
%\abs{x \cdot y}^2= (x \cdot y)^2= x^2 \cdot y^2= \abs{x}^2 \cdot \abs{y}^2= (\abs{x} \cdot \abs{y})^2 \ .
%\end{equation*}
 %Como $\abs{x \cdot y} \geqslant 0$ e $\abs{x} \cdot \abs{y} \geqslant 0$ resulta
%\begin{equation*}
%\abs{x \cdot y}= \abs{x} \cdot \abs{y} \ . 
%\end{equation*}

 \item $\abs{\dfrac{x}{y}}= \dfrac{\abs{x}}{\abs{y}}$, para $y \neq 0$;%, de fato,
%\begin{equation*}
%\abs{\dfrac{x}{y}}^2= \left(\dfrac{x}{y} \right)^2= \dfrac{x^2}{y^2}= \dfrac{\abs{x}^2}{\abs{y}^2}= \left( \dfrac{\abs{x}}{\abs{y}} \right)^2 \ . 
%\end{equation*}
% Como $\abs{\dfrac{x}{y}} \geq 0$ e $\dfrac{\abs{x}}{\abs{y}} \geq 0$ resulta que
%\begin{equation*}
%\abs{\dfrac{x}{y}}= \dfrac{\abs{x}}{\abs{y}} \ .
%\end{equation*}

 \item \emph{Desigualdade triangular}:\\ $\abs{x+y}\leq \abs{x}+\abs{y}$; %, de fato, \\
 %se $x + y \geqslant 0$, pela definição de módulo, $\abs{x+y}= x+y \leq \abs{x} + \abs{y}$; \\
 %se $x + y < 0$, pela definição de módulo, $\abs{x+y}= -(x+y)= -x-y \leq \abs{x} + \abs{y}$. \\
 %Logo, para quaisquer $x, y \in \R$ temos que
%\begin{equation*}
%\abs{x+y} \leq \abs{x}+\abs{y} \ .
%\end{equation*}

 \item $\abs{x-y} \leq \abs{x} + \abs{y}$;%, de fato \\
 %Note que $x-y= x+ (-y)$, logo $\abs{x-y}= \abs{x+ (-y)}$ aplicando a desigualdade trinagular temos,
%\begin{equation*}
%\abs{x-y}= \abs{x+ (-y)} \leq \abs{x} + \abs{-y}= \abs{x} + \abs{y} \ .
%\end{equation*}

 \item $\abs{\abs{x} - \abs{y}} \leq \abs{x - y}$%, para mostrar esta desigualdade vamos fazer por partes.
 %\begin{itemize}
 %\item $\abs{x} - \abs{y} \leq \abs{x - y}$, \\
 %de fato, pela desigualdade triangular temos que
%\begin{equation*}
%\abs{z+y} \leq \abs{z} + \abs{y}
%\end{equation*}
% subtraíndo $\abs{y}$ a ambos os termos temos,
%\begin{equation*}
%\abs{z+y} - \abs{y} \leq \abs{z}
%\end{equation*}
% fazendo $x= z+y$ temos que $z=x-y$ substituindo estes valores na equação acima obtemos
%\begin{equation*}
%\abs{x} - \abs{y} \leq \abs{x-y} \ . 
%\end{equation*}
 \item $\abs{y} - \abs{x} \leq \abs{x - y}$.%, \\
 %de fato, pela desigualdade triangular temos que
%\begin{equation*}
%\abs{x+z} \leq \abs{x} + \abs{z}
%\end{equation*}
% subtraíndo $\abs{x}$ a ambos os termos temos,
%\begin{equation*}
%\abs{x+z} - \abs{x} \leq \abs{z}
%\end{equation*}
 %fazendo $y= x+z$ temos que $z=y-x$ substituindo estes valores na equação acima obtemos
%\begin{equation*}
%\abs{y} - \abs{x} \leq \abs{y-x}= \abs{x-y} \ . 
%\end{equation*}
%\end{itemize}

% Portanto,
%\begin{equation*}
% \abs{\abs{x} - \abs{y}} = \pm (\abs{x} - \abs{y}) \leq \abs{x-y} \ .
%\end{equation*}
\end{multicols}
\end{enumerate}
\end{prop}

 Você pode se deparar, em alguns momentos, com expressões matemáticas que envolvem módulo. Elas são chamadas de expressões matemáticas modulares. É preciso entender que operar com módulo pode ser algo bastante complicado e, para fugir disso, o que se faz, na prática, é ``abrir o módulo'' usando a definição dada pela Equação \ref{eq:modulo}. Vejamos os exemplos a seguir.

 \begin{exem}
  Reescreva a expressão $\dfrac{\abs{x}}{x}$ eliminando o módulo.

  Primeiramente, observe que devemos considerar $x \neq 0$, pois a expressão apresenta uma divisão por $x$ (e não podemos dividir por 0). Relembrando a definição de $|x|$ dada na Equação \ref{eq:modulo}, concluímos que os casos a serem analisados são: $x<0$ e $x>0$.

  \begin{itemize}
   \item Caso $x> 0$. Para este caso temos $\abs{x}= x$ e a expressão pode ser reescrita como
\begin{equation*}
\dfrac{\abs{x}}{x}=\dfrac{x}{x}= 1.
\end{equation*}
   \item Caso $x< 0$. Para este caso, temos $\abs{x}= -x$ e a expressão pode ser reescrita como
\begin{equation*}
\dfrac{\abs{x}}{x}=\dfrac{-x}{x}= -1.
\end{equation*}
  \end{itemize}

 Portanto, podemos reescrever a expressão $\dfrac{\abs{x}}{x}$ da seguinte maneira:

 \begin{equation*}
   \frac{\abs{x}}{x} = \left\{ \begin{array}{rl}
        -1, & \text{se} \ \ x< 0 \\
         1, & \text{se } \ \ x> 0
     \end{array}\right..
 \end{equation*}
\end{exem}
 
 \begin{exem}
Reescreva a expressão modular $\abs{-5x^5}$ eliminando o módulo.

Usando as propriedades e a definição de módulo podemos escrever
\begin{equation*}
\abs{-5x^5}= \abs{-5}\cdot \abs{x^5}= 5 \cdot \abs{x^5} = \left\{ \begin{array}{rl}
        5 x^5, & \text{se} \ \ x^5\geq 0 \\
         5 (-x^5), & \text{se } \ \ x^5< 0
     \end{array}\right.
     = \left\{ \begin{array}{rl}
        5 x^5, & \text{se} \ \ x\geq 0 \\
         -5x^5, & \text{se } \ \ x< 0
     \end{array}\right..
\end{equation*}
 Observe que, como o expoente é ímpar, devemos considerar a possibilidade de $x^5$ ser um número positivo ou negativo (e isso está diretamente ligado ao fato de $x$ ser positivo ou negativo).
 \end{exem}
 
 \begin{exem}
  $\abs{-5x^4}$
\begin{equation*}
\abs{-5x^4}= \abs{-5}\cdot \abs{x^4}= \abs{-5}\cdot \abs{x}^4 = 5 \cdot x^4.
\end{equation*}

Observe que foram usadas as propriedades 5 ($\abs{-5} = 5$) e 7 ($\abs{x^4} = \abs{x}^4$). E como o expoente é par, o valor de $x^4$ é positivo, independente do sinal de $x$. Neste caso, $\abs{x^4} = x^4$. 
 \end{exem}
 
 \begin{exem}
  Simplifique a expressão $\abs{\dfrac{2x^2y}{4xy^3}}$ usando propriedades de módulo:
\begin{equation*}
\abs{\dfrac{2x^2y}{4xy^3}} = \abs{\dfrac{x}{2y^2}}= \dfrac{\abs{x}}{\abs{2y^2}}= \dfrac{\abs{x}}{\abs{2} \cdot \abs{y^2}}= \frac{\abs{x}}{2 \cdot y^2}.
\end{equation*}
 \end{exem}
 
 \begin{exem}
 Simplifique a expressão  $\dfrac{\abs{-6x}}{5} - \abs{\dfrac{-3x}{2}}$ usando propriedades de módulo:

 \begin{eqnarray*}
  \dfrac{\abs{-6x}}{5} - \abs{\dfrac{-3x}{2}} &=&
 \dfrac{\abs{-6}\cdot \abs{x}}{5} - \dfrac{\abs{-3} \cdot \abs{x}}{\abs{2}} \\
 &=& \dfrac{6 \cdot \abs{x}}{5} - \dfrac{ 3 \cdot \abs{x}}{2} \\
 &=& \dfrac{12 \cdot \abs{x}}{10} - \dfrac{15 \cdot \abs{x}}{10} \\
 &=& \dfrac{12\abs{x} - 15\abs{x}}{10} \\
 &=& \dfrac{-3\abs{x}}{10}.
 \end{eqnarray*}

\end{exem}

 \begin{exem}
  Simplifique a expressão $\abs{x-2} + \abs{x+4}$ \label{eqmodulo}

 Para simplificar esta expressão, primeiro precisamos usar a definição de módulo para cada um dos termos que estão sendo somados:
 \begin{align*}
    \abs{x-2} = \left\{\begin{array}{rl}
      -(x-2), & \text{se} \ \ x-2< 0 \\
      x-2, & \text{se } \ \ x-2 \geq 0
     \end{array} \right.
     &\Rightarrow
     \abs{x-2} =  \left\{\begin{array}{rl}
      -x + 2, & \text{se} \ \ x < 2 \\
      x-2, & \text{se } \ \ x \geq 2
     \end{array}\right.\\
      \abs{x+4} =  \left\{\begin{array}{rl}
      -(x+4), &\text{se} \ \ x+4< 0 \\
      x+4, & \text{se } \ \ x+4 \geq 0
     \end{array}\right.
     &\Rightarrow
     \abs{x+4} =  \left\{\begin{array}{rl}
      -x - 4, & \text{se} \ \ x < -4 \\
      x + 4, & \text{se } \ \ x \geq -4
     \end{array}\right.
  \end{align*}

  Observe que $(x-2)$ muda de sinal quando $x=2$, e que $(x+4)$ muda de sinal quando $x=-4$, logo esta soma de módulos tem uma definição particular para cada um dos intervalos $(-\infty, -4)$, $[-4, 2)$ e $[2, \infty)$. Para facilitar a compreensão organizamos na tabela seguinte a soma, em cada caso.

   \begin{table}[H]
 \centering
 \begin{tabular}{|c|c|c|c|} \hline
 \rowcolor{gray}
  Expressão & $(-\infty, -4)$ & $[-4, 2)$ & $[2, \infty)$  \\\hline
  $\abs{x-2}$ & $-x+2$ &  $-x+2$ & $x-2$ \\\hline
  $\abs{x+4}$ & $-x-4$ &  $x+4$ & $x+4$ \\\hline
  $\abs{x-2} + \abs{x+4}$ & $-x+2-x-4$ & $-x+2+x+4$ & $x-2+x+4$ \\\hline
 \end{tabular}
\end{table}

 Portanto, temos que
 \begin{equation*}
 \abs{x-2} + \abs{x+4} =  \left\{ \begin{array}{rl}
      -2x-2, & \text{se} \ \ x < -4 \\
      6, &\text{se } \ \ -4 \leq x < 2 \\
      2x + 2, & \text{se} \ \ x \geq 2 
     \end{array}\right..
 \end{equation*} 
 \end{exem}


 \section{Equações modulares}

\begin{obs}
  As equações modulares são equações que apresentam expressões dentro de um módulo. Em particular, vamos analisar casos em que esta expressão matemática é um polinômio. Por exemplo, temos
  \[\abs{p(x)}= 0.\]

  Assim como nas equações de 1º e 2º grau vistas no Capítulo 5, resolver uma equação modular significa encontrar todos os valores de $x$ que tornam a equação verdadeira (satisfazem a equação), considerando um conjunto universo específico (por exemplo, os reais).
 \end{obs}

 Antes de começarmos a ver exemplos destas equações lembremos que para um número real $x$ qualquer:

 \[
\abs{x}= \left\{\begin{array}{rl}
      -x , & \text{se} \ \ x<0 \\
      x , & \text{se} \ \ x \geq 0
     \end{array}\right. .
\]

Além disso, lembre-se que todas as propriedades listadas em \ref{prop.modulo} são válidas e podem ser usadas durante a resolução de uma equação modular.

Os exemplos a seguir resolvem equações modulares para um melhor entendimento.

\begin{exem} 
  Suponha que $a> 0$. Resolva a equação $\abs{x}= a$.

Vamos analisar a equação considerando os casos $x\geq 0$ e $x < 0$.

Para $x\geq 0$, temos $\abs{x} = x = a$.

Para $x<0$, temos $\abs{x} = -x = a$, ou seja, $x = -a$.

Portanto o conjunto solução desta equação é $S= \left\{-a, a \right\}$.
\end{exem}

\begin{exem}
  Resolva $\abs{x}= 10$ (observe que este é um caso particular do exemplo anterior).

 Novamente, vamos analisar a equação considerando os casos $x\geq 0$ e $x < 0$.

Para $x\geq 0$, temos $\abs{x} = x = 10$.

Para $x<0$, temos $\abs{x} = -x = 10$, ou seja, $x = -10$.

Portanto o conjunto solução desta equação é $S= \left\{-10, 10 \right\}$.
\end{exem}

\begin{exem}
 Resolva a equação $\abs{2x-2}= 10$.

Vamos analisar a equação considerando os casos $2x - 2 \geq 0$ e $2x - 2 < 0$, ou seja, $x \geq 1$ e $x < 1$.

Para $x\geq 1$ temos $\abs{2x-2} = 2x-2$. Assim, $$\abs{2x-2} = 10 \Leftrightarrow 2x - 2 = 10 \Leftrightarrow 2x = 12 \Leftrightarrow x = 6.$$

Para $x < 1$ temos $\abs{2x-2} = -(2x-2)= -2x+2$. Assim, $$\abs{2x-2} = 10 \Leftrightarrow -2x + 2 = 10 \Leftrightarrow -2x = 8 \Leftrightarrow x = -4.$$

Portanto, o conjunto solução desta equação é $S= \left\{-4, 6 \right\}$.
 \end{exem}
 
 \begin{exem}
 Resolva $\abs{2x^2 - 72}= 26$.

Observe que, aplicando a definição de módulo, podemos reescrever o lado esquerdo da equação como:

\begin{equation}
\label{ex:mod_quadratico}
\abs{2x^2 - 72} = \left\{\begin{array}{rl}
2x^2 - 72, & \mbox{se} \; \;  2x^2 - 72 \geq 0\\
-(2x^2 - 72), & \mbox{se} \; \; 2x^2 - 72 < 0
\end{array}\right. \end{equation}

Desse modo, devemos fazer um estudo de sinal e decidir os intervalos em que $2x^2 - 72$ é positivo, negativo ou nulo (se ainda tem dúvida em como fazer isso, consulte o Capítulo 6). Para este estudo, reescrevemos $2x^2 - 72$ em sua forma fatorada, ou seja, $$2x^2 - 72 = 2(x+6)(x-6) = (2x+ 12)(x-6)$$.

\begin{signtbl}{3}{2}
        \linelabel{1}{$2x+12$}
        %\numbernode{6}{1}{2}
        \numbernode{-6}{1}{1}
        \numbernode{6}{3}{2}
        \signs{1}{-1,1,1}
        \linelabel{2}{$x-6$}
        %\numbernode{-6}{2}{1}
        \numbernode{6}{2}{2}
        \numbernode{-6}{3}{1}
        \signs{2}{-1,-1,+1}
        \intervalcolor{violet}
        \signs{3}{1,-1,+1}
        \linelabel{3}{$(2x+12)(x-6)$}
        \intervalsign{3}{02}{1}
        \intervalsign{3}{20}{2}
\end{signtbl}

Resumindo, a análise de sinal nos permite avaliar que:

     \[
     \begin{cases}
      2x^2- 72 \geq 0, \ \ \text{se} \ \ x \leq -6 \; \; \mbox{ou} \; \; x \geq 6\\
      2x^2- 72 < 0, \ \ \text{se} \ \ -6 < x < 6
      \end{cases}
     \]
Podemos então reescrever a Expressão \ref{ex:mod_quadratico} como:

\begin{equation}
\abs{2x^2 - 72} = \left\{\begin{array}{rl}
2x^2 - 72, & \mbox{se} \; \;  2x^2 - 72 \geq 0\\
-(2x^2 - 72), & \mbox{se} \; \; 2x^2 - 72 < 0
\end{array}\right. = \left\{\begin{array}{rl}
2x^2 - 72, & \mbox{se} \; \;  x \leq -6 \; \; \mbox{ou} \; \; x \geq 6\\
-2x^2 + 72, & \mbox{se} \; \; -6 < x < 6
\end{array}\right. \end{equation}

Temos, portanto, duas equações para resolver separadamente.

Se $x\leq 6$ ou $x \geq 6$ então
\begin{equation*}
2x^2- 72= 26 \Leftrightarrow 2x^2= 98 \Leftrightarrow x^2= 49 \Leftrightarrow x= - 7\; \mbox{ou} \; \; x = 7.
\end{equation*}

     Note que as duas soluções pertencem ao intervalo que estamos considerando a análise ($x = 7$ está no intervalo $x \geq 6$ e $x = -7$ está no intervalo $x\leq -6$) e, portanto, elas são de fato soluções da equação.

Se $-6 < x < 6$ então
      \begin{eqnarray*}
      -(2x^2-72)= 26 \Leftrightarrow -2x^2 + 72= 26 \Leftrightarrow -2x^2= -46 \\
     \Leftrightarrow x^2= 23 \Leftrightarrow x= -\sqrt{23} \; \; \mbox{ou} \; \;  x = \sqrt{23}.
      \end{eqnarray*}

Note que $\sqrt{23} \approx 4,78$ e que, portanto, temos $-6 < -\sqrt{23} < \sqrt{23} < 6$, ou seja, ambas as soluções encontradas pertencem ao intervalo de análise e podem ser consideradas. Juntando as análises intervalares realizadas podemos concluir que o conjunto solução da equação modular $\abs{2x^2 - 72|}$ é 
\begin{equation*}
S= \{-7, -\sqrt{23}, \sqrt{23}, 7\}.
\end{equation*}
\end{exem}

\begin{exem}
 \colorbox{yellow}{Atenção! O cojunto solução de uma equação modular pode ser vazio.} Resolva $\abs{x-4}=-2$.

Perceba que, pela Propriedade 1 podemos afirmar que o módulo de qualquer número é sempre maior ou igual a zero. Desse modo, não há valor de $x$ que torne $\abs{x-4} = -2$. Só com essa análise podemos concluir que o conjunto solução é $S = \{  \}$. Ou podemos apenas dizer que não há solução para esta equação.

Alguém mais curioso poderia se questionar se não há como encontrar o conjunto vazio como solução de maneira algébrica. Se este for o caso, faríamos a seguinte resolução:

Pela definição de módulo temos que
     \[
    \abs{x-4} = \left\{\begin{array}{rl}
    -(x-4), & \text{se} \ \ x-4 < 0 \\
    x-4, & \text{se } \ \ x-4 \geq 0
     \end{array}\right.
     =
    \left\{\begin{array}{rl}
    -x+4, & \text{se} \ \ x < 4 \\
    x-4, & \text{se } \ \ x \geq 4
     \end{array}\right..
     \]
     Então, 
     \[
     \abs{x-4} = -2 \Leftrightarrow \left\{\begin{array}{rl}
    -x+4=-2, & \text{se} \ \ x < 4 \\
    x-4 = -2, & \text{se } \ \ x \geq 4
     \end{array}\right.
     \Leftrightarrow \left\{\begin{array}{rl}
     x = 6, & \text{se} \ \ x < 4 \\
    x = 2, & \text{se } \ \ x \geq 4
     \end{array}\right.
     \]
 Observe que, na condição $x<4$ encontramos $x = 6$, ou seja, não há solução neste intervalo. De modo análgo, encontramos a solução $x = 2$ para o intervalo $x \geq 4$, o que não faz sentido. Concluímos, portanto, que a equação não tem solução (ou que o conjunto solução é vazio).
\end{exem}

 Vamos agora colocar, de forma mais geral, o que acabamos de ver nos exemplos. A vantagem de entender este processo de forma geral é que você poderá usá-lo na resolução de qualquer equação modular. 
 
 Consideremos uma equação modular na forma geral
\begin{equation*}
\abs{A}= B,
\end{equation*}
 em que $A$ e $B$ são expressões algébricas quaisquer. Pela definição de módulo temos que
 \[ \abs{A}= \begin{cases}
          -A, \ \text{se} \ \ A < 0 \\
           A, \ \text{se} \ \ A \geq 0
         \end{cases}.
 \]
 Logo, as soluções da equação modular devem satisfazer
\begin{equation*}
A= B \ \ \ \text{ou} \ \ \ -A= B \ .
\end{equation*}
 Além disso, lembremos que, por propriedade de módulo, $\abs{A}\geq 0$, independente da expressão $A$.
 
 Como nossa equação é da forma  $\abs{A}= B$, logo garantir $\abs{A}\geq 0$ é equivalente a garantir $B \geq 0$, donde obtemos que as equações com $B < 0$ não possuem solução.

 Resumindo, temos que uma variável $x$ é solução da equação modular $\abs{A}= B$ se ela satisfizer:
\begin{equation*}
(A= B \ \ \ \text{ou} \ \ \ -A= B) \ \ \ \text{e} \ \ \ (B \geq 0) \ . 
\end{equation*}

 Vejamos mais um exemplo resolvido de equação modular.

 \begin{exem}
   Resolva $\abs{x-2} + \abs{x+4}= 10$.

   Observe que temos $A = \abs{x-2} + \abs{x+4}$ e $B = 10$. Como $B > 0$, basta analisar os casos $A= B$ e $-A= B$.

   $A$ é composto pela soma de duas expressões modulares vistas no Exemplo \ref{eqmodulo}. Ao abrirmos o módulo, obtivemos a expressão 
   \[ \abs{x-2} + \abs{x+4} = \left\{\begin{array}{rl}
      -2x-2, & \text{se} \ \ x < -4 \\
      6, & \text{se } \ \ -4 \leq x < 2 \\
      2x + 2, & \text{se} \ \ x \geq 2 \ .
     \end{array}\right..
  \]
 Para resolver a equação modular  dada faremos a análise para cada intervalo.

 \textbf{Intervalo 1:} $x < -4$

 Neste caso, $\abs{x-2} + \abs{x+4}= -2x-2$, ou seja, 
\begin{equation*}
-2x-2= 10 \Leftrightarrow -2x= 12 \Leftrightarrow x= -6.
\end{equation*}
 Como $x= -6$ pertence ao intervalo $x< -4$, ela é uma solução a ser considerada.

 \textbf{Intervalo 2:} $-4 \leq x < 2$

 Neste caso, $\abs{x-2} + \abs{x+4}= 6$, e nossa equação se torna $6= 10$. Isto é impossível, donde concluímos que neste intervalo a equação não tem solução (parece estranho? Isto significa que não há nenhum número real entre -4 e 2 que torna a expressão $\abs{x-2} + \abs{x+4}$ igual a 6).

 \textbf{Intervalo 3:} $x \geq 2 $

 Neste caso, $\abs{x-2} + \abs{x+4}= 2x + 2$ e, então,
\begin{equation*}
2x + 2= 10 \Rightarrow 2x= 8 \Rightarrow x= 4.
\end{equation*}

Como $x= 4$ pertence ao intervalo $x \geq 2$, ela será uma solução a ser considerada.

 Portanto, após a análise das soluções nos três intervalos concluímos $S=\{-6, 4\}$ como  o conjunto solução da equação modular
$\abs{x-2} + \abs{x+4}= 10.$
 \end{exem}


 \section{Inequações modulares}

 Nesta seção usaremos as propriedades da ordem do conjunto dos números reais e também as propriedades de módulo de um número real. Neste momento já supomos que o leitor está bem familiarizado com as tais propriedades.%, mas na dúvida você pode consultar as propriedades que estão listadas nas %\ref{prop.ordem} e \ref{prop.modulo}.

Vejamos a seguir alguns exemplos de inequações modulares.

 \begin{exem} 
  Suponha que $a> 0$. Resolva a inequação $\abs{x} < a$.

Faremos esta resolução de modo semelhante ao que fizemos para a resolução das equações modulares, levando em conta, agora, o sinal da desigualdade.

Mais uma vez, lembre-se que 
 \[
\abs{x}= \left\{\begin{array}{rl}
      -x , & \text{se} \ \ x<0 \\
      x , & \text{se} \ \ x \geq 0
     \end{array}\right. .
\]

Para resolver a inequação, vamos analisar os dois intervalos que aparece ao abrirmos o módulo.

\textbf{Intervalo 1:} $x < 0$

Neste caso temos a desigualdade $-x < a$ e, portanto, $x > -a$. Como estamos procurando soluções na região onde $x$ é estritamente negativo, nos limitaremos aos valores que satisfaçam as duas condições: $x < 0$ e $x > -a$. Ou seja, neste intervalo temos como solução o conjunto $S_1 = \{x \in \R | \ -a < x < 0\}$.

\textbf{Intervalo 2:} $x\geq 0$

Neste caso temos a desigualdade $x<a$, que já se resolve de forma direta. Perceba que, como nossa região de análise são os números estritamente positivos, a solução deve considerar ambas as desigualdades ($x \geq 0$ e $x < a$). Ou seja, neste intervalo temos como solução o conjunto $S_2 = \{x \in \R | \ 0 \leq x < a\}$.

Assim, o conjunto solução da inequação dada é o conjunto constituído pelas soluções encontradas no Intervalo 1 unido com as soluções encontradas no Intervalo 2. Faremos então uma união de conjuntos e, portanto, $S = S_1 \cup S_2 = \{x \in \R| -a < x < a\} = (-a,a)$.

\begin{comment}
\begin{equation*}
\abs{x} < a \Leftrightarrow \abs{x}^2 < a^2 \Leftrightarrow x^2 < a^2
\end{equation*}
   mas,
\begin{equation*}
x^2 < a^2 \Leftrightarrow x^2 - a^2 < 0 \Leftrightarrow (x-a)(x+a) < 0 \ .
\end{equation*}
   Vamos então analisar quando $(x-a)(x+a) < 0$. Lembremos que produto de dois números é negativo quando um deles for negativo e o outro positivo, com isso a inequação $(x-a)(x+a) < 0$ é satisfeita em duas situações:\\
   Caso 1: $x-a<0$ e $x+a>0$
\begin{equation*}
x-a<0 \Rightarrow x< a
\end{equation*}
   e
\begin{equation*}
x+a>0 \Rightarrow x>-a \Rightarrow -a< x
\end{equation*}
  Fazendo a interseção dos conjuntos $A_1= \left\{ x \in \R \mid x<a \right\}$ e $B_1= \left\{ x \in \R \mid -a<x \right\}$, obtemos $A_1 \cap B_1= \left\{ x \in \R \mid -a< x <a \right\}$. O conjunto $A_1 \cap B_1$ é o conjunto solução da inequação neste caso.


   Caso 2: $x-a>0$ e $x+a<0$
\begin{equation*}
x-a> 0 \Rightarrow x> a
\end{equation*}
   e
\begin{equation*}
x+a< 0 \Rightarrow x<-a
\end{equation*}
   Fazendo a interseção dos conjuntos $A_2= \left\{ x \in \R \mid x> a \right\}$ e $B_2= \left\{ x \in \R \mid x<-a \right\}$, obtemos $A_2 \cap B_2= \emptyset$. Portanto neste caso a inequação não possui solução.

  Portanto $\abs{x} < a \Leftrightarrow -a< x <a$.
\end{comment}
  \end{exem}
  
\begin{exem}
   Caso particular: Resolva $\abs{x} \leq 5$.
\begin{equation*}
\abs{x} \leq 5 \Leftrightarrow -5 \leq x \leq 5.
\end{equation*}
  Logo, o conjunto solução desta inequação é
\begin{equation*}
S=\{x \in \R| -5 \leq x \leq 5\} = [-5, 5]. 
\end{equation*}
\end{exem}

\begin{exem}
   Suponha que $a> 0$. Resolva a inequação $\abs{x} > a$.

Faremos esta resolução de modo semelhante ao que fizemos para a resolução das equações modulares, levando em conta, agora, o sinal da desigualdade.

Mais uma vez, lembre-se que 
 \[
\abs{x}= \left\{\begin{array}{rl}
      -x , & \text{se} \ \ x<0 \\
      x , & \text{se} \ \ x \geq 0
     \end{array}\right. .
\]

Para resolver a inequação, vamos analisar os dois intervalos que aparece ao abrirmos o módulo.

\textbf{Intervalo 1:} $x < 0$

Neste caso temos a desigualdade $-x > a$ e, portanto, $x < -a$. Como estamos procurando soluções na região onde $x$ é estritamente negativo, nos limitaremos aos valores que satisfaçam as duas condições: $x < 0$ e $x < -a$. Ou seja, neste intervalo temos como solução o conjunto $S_1 = \{x \in \R | x < -a\} = (-\infty, -a)$.

\textbf{Intervalo 2:} $x\geq 0$

Neste caso temos a desigualdade $x>a$, que já se resolve de forma direta. Perceba que, como nossa região de análise são os números estritamente positivos, a solução deve considerar ambas as desigualdades ($x \geq 0$ e $x > a$). Ou seja, neste intervalo temos como solução o conjunto $S_2 = \{x \in \R | x > a\} = (a, +\infty)$.

Assim, o conjunto solução da inequação dada é o conjunto constituído pelas soluções encontradas no Intervalo 1 unido com as soluções encontradas no Intervalo 2. Faremos então uma união de conjuntos e, portanto, $S = S_1 \cup S_2 = \{x \in \R| x < -a \ \ \mbox{ou} \ \ x > a\} = (-\infty,a) \cup (a, + \infty)$.

\begin{comment}
\begin{equation*}
\abs{x} > a \Leftrightarrow \abs{x}^2 > a^2 \Leftrightarrow x^2 > a^2
\end{equation*}
   mas,
\begin{equation*}
x^2 > a^2 \Leftrightarrow x^2 - a^2 > 0 \Leftrightarrow (x-a)(x+a) > 0 \ .
\end{equation*}
   Vamos então analisar quando $(x-a)(x+a) > 0$. Lembremos que produto de dois números é positivo quando eles tem o mesmo sinal, com isso a inequação $(x-a)(x+a) > 0$ é satisfeita em duas situações:\\
   Caso 1: $x-a< 0$ e $x+a< 0$
\begin{equation*}
x-a< 0 \Rightarrow x< a
\end{equation*}
   e
\begin{equation*}
x+a< 0 \Rightarrow x< -a \ .
\end{equation*}
  Fazendo a interseção dos conjuntos $A_1= \left\{ x \in \R \mid x<a \right\}$ e $B_1= \left\{ x \in \R \mid x< -a \right\}$, obtemos $A_1 \cap B_1= \left\{ x \in \R \mid x< -a \right\}$. O conjunto $A_1 \cap B_1$ é o conjunto solução da inequação neste caso.


   Caso 2: $x-a> 0$ e $x+a> 0$
\begin{equation*}
x-a> 0 \Rightarrow x> a
\end{equation*}
   e
\begin{equation*}
x+a> 0 \Rightarrow x> -a 
\end{equation*}
   Fazendo a interseção dos conjuntos $A_2= \left\{ x \in \R \mid x> a \right\}$ e $B_2= \left\{ x \in \R \mid x> -a \right\}$, obtemos $A_2 \cap B_2= \left\{ x \in \R \mid x> a \right\}$. O conjunto $A_2 \cap B_2$ é o conjunto solução da inequação neste caso.

   Agora fazendo $S= (A_1 \cap B_1) \cup (A_2 \cap B_2)$ obtemos que $S= \left\{ x \in \R \mid x<-a \text{ ou } x> a \right\}$ é o conjunto solução da inequação $\abs{x} > a$.

  Portanto $\abs{x} > a \Leftrightarrow x<-a \text{ ou } x> a$.
\end{comment}
\end{exem}

\begin{exem}
   Caso particular: Resolva $\abs{x} \geq 5$.

Pelo caso estudado anteriormente podemos afirmar que \begin{equation*}
\abs{x} \geq 5 \Leftrightarrow x \leq -5 \ \ \text{ ou } \ \ x \geq 5.
\end{equation*}
   
Logo, o conjunto solução desta inequação é:
\begin{equation*}
S= \{x \in \R| x \leq -5 \ \ \mbox{ou} \ \ x > 5\}= (-\infty, -5] \cup [5, \infty) \ . 
\end{equation*}
 \end{exem}
 
%  \section{Equações radicais ou irracionais}
%  \colorbox{blue}{
%  \begin{minipage}{0.9\linewidth}
%  \begin{center}
%   As equações radicais são equações nas quais temos polinômios dentro de raízes. Como por exemplo:
%   \[\sqrt[n]{p(x)}= 0\]
%   onde $n \in \N$ e $p(x)$ é um polinômio na variável $x$. Caso $n \in \N$ seja par precisamos ainda ter $p(x) \geqslant 0$.
%  \end{center}
%  \end{minipage}}
% 
%%  Lembre-se que não existe raíz de ordem par de número negativo, logo a equação radical
%  \[\sqrt[n]{p(x)}= 0\]
%  caso $n \in \N$ seja par, está definida apenas no conjunto $D= \{x \in \R \mid p(x) \geqslant 0\}$.
% 
%%  Este subconjunto dos números reais no qual a equação está bem definida é chamado de domínio da equação. A solução de uma equação radical é necessariamente um subconjunto do domínio da mesma.
% 
%%  \begin{exem}
%  $\dfrac{1}{\sqrt{x}}= 4$
% 
%%  Para resolver esta equação começamos determinando seu domínio. Note que aqui precisamos ter duas condições sendo satisfeitas:
%  \begin{enumerate}[i)]
%  \item $\sqrt{x} \neq 0 \Rightarrow x \neq 0$;
%  \item $x \geqslant 0$ pois temos uma raíz de ordem $2$.
%  \end{enumerate}
%  O domínio será portanto a interseção dos dois conjuntos acima que é dado por:
%  \[D= \{ x \in \R \mid x > 0 \} \ . \]
% 
%%  Vamos agora resolver a equação, 
%  \[\dfrac{1}{\sqrt{x}}= 4 \Rightarrow \left(\dfrac{1}{\sqrt{x}}\right)^2= 4^2 \Rightarrow \dfrac{1}{x}= 16 \Rightarrow \dfrac{1}{16}= x\]
% 
%%  Como $\dfrac{1}{16} \in D$ esta é de fato a solução da equação.
%  \end{exem}
% 
%  \section{Inequações radicais ou irracionais}
% 
%   \vskip0.3cm
%  \colorbox{blue}{
%  \begin{minipage}{0.9\linewidth}
%  \begin{center}
%   As inequações radicais são inequações nas quais temos polinômios dentro de raízes. Estas inequações possuem domínios com as mesmas restrições das equações radicais.
%  \end{center}
%  \end{minipage}}
%  \vskip0.3cm
 
\begin{secExercicios}

 \begin{exer}
    Simplifique as expressões, eliminando o módulo:
    \begin{enumerate}[a)] \begin{multicols}{2}
        \item $\dfrac{|x-1|}{x-1}$
        \item $\dfrac{|x+3|}{x+3}$
        \item $\dfrac{|x|}{x}+\dfrac{|x-4|}{x-4}$
        \item $\dfrac{\abs{x-3}}{x-3}$
        \item $1 + \dfrac{\abs{x-2}}{x-2}$
        \item $\dfrac{\abs{x}}{x}, \; 0 < x < 4$
       \end{multicols} \end{enumerate}
\end{exer}

 \begin{exer}
    Resolva as desigualdades, em cada caso:
    \begin{enumerate}[a)]
        \item $\abs{x} < 1$
        \item $\abs{1-5x} <4$
        \item $\abs{7x-4} <10$
        \columnbreak
        \item $\abs{x-2} < -2.\abs{-x}$
        \item $\abs{3+9x} <1$
        \item $\abs{3x+4}\leq 2$
        \item $\abs{4x-4} \geq 2$
        \item $\abs{x+5} >2$
        \item $\abs{2-4x} \geq 3$
        \item $\abs{x-3} > -1$
    \end{enumerate}
\end{exer}


\begin{exer}
    Resolva a equação dada:
    \begin{enumerate}[a)]
        \item $\abs{x} = 1248$
        \item $\abs{x} = -2$
        \item $\abs{x-5} = 4$
        \item $\abs{x}= 2.\abs{-x}$
        \item $\abs{2x-4} = 6$
        \item $\abs{3x+1}=\abs{x-2}$
        \item $\abs{6-3x} = 10$
        \item $\abs{(x-1)(x+2)} = 0$
    \end{enumerate}
\end{exer}


\begin{exer}
    Determine o conjunto solução das seguintes equações:
    \begin{enumerate}[a)]
        \item $\abs{\dfrac{x-2}{3}} =5$
        \item $\abs{\dfrac{1-x}{4}}=6$
        \item $\abs{2x-3} = \abs{4x+5}$
        \item $\abs{5x-4} = \abs{3x+6}$
        \item $\abs{2x+5} = \abs{x-11}$
        \item $\abs{x-3} + \abs{x+4} = 7$
        \item $\abs{x-3} + \abs{x-4}=1$
        \item $\abs{\abs{x-5} -8} = 6$
    \end{enumerate}
\end{exer}





\begin{exer}
    As afirmações abaixo são verdadeiras ou falsas? Justifique sua resposta.
    \begin{enumerate}[a)]
        \item $\abs{a}$ é sempre um número positivo.
        \item $\abs{a}$ pode ser um número negativo.
        \item $\abs{a} = a$, para todo número $a$ real.
        \item $\abs{x}$ pode ser zero.
        \item $\abs{a.b.c} = \abs{a}.\abs{b}.\abs{c}$
        \item $\abs{a.b} < \abs{a}\abs{b}$, para quaisquer $a,b$ reais.
        \item $\abs{a + b} = \abs{a} + \abs{b}$, para quaisquer $a, b$ reais.
        \item $\abs{-2 + c} = 2 + \abs{c}$, para todo $c \leq 0$.
     \end{enumerate}
\end{exer}

\end{secExercicios}

%\subsection*{Respostas:}

%\shipoutAnswer
