%----------------------------------------------------------------------------------------
%	TITLE PAGE
%----------------------------------------------------------------------------------------

\titlepage % Output the title page
	{\includegraphics[width=\paperwidth]{background.pdf}} % Code to output the background image, which should be the same dimensions as the paper to fill the page entirely; leave empty for no background image
	{ % Title(s) and author(s)
		\centering\sffamily % Font styling
        \includegraphics[width=2.3cm]{./Images/logouff_vertical_azul.png}
        
		\vspace{16pt} % Vertical whitespace
		{\Huge\bfseries Introdução à Matemática Superior\par} % Book title
		\vspace{10pt} % Vertical whitespace
		{\LARGE Dep. Matemática -- ICEx -- Volta Redonda\par} % Subtitle
        \vspace{16pt} % Vertical whitespace
  
		{\huge\bfseries Jordan Lambert \par} % Author name
        \vspace{5pt}
        {Versão: \today\par}
	}



%\listoffigures % Output the list of figures, comment or remove this command if not required

%\listoftables % Output the list of tables, comment or remove this command if not required

%----------------------------------------------------------------------------------------
%	COPYRIGHT PAGE
%----------------------------------------------------------------------------------------

\thispagestyle{empty} % Suppress headers and footers on this page

~\vfill % Push the text down to the bottom of the page

\noindent Por Jordan Lambert.

\vspace{.5cm}
\noindent Elaborado a partir material colaborativo ``Pré-Cálculo'' organizado por Francieli Triche (UFSC) e Helder Geovane Gomes de Lima (UFSC), com Licença Creative Commons Atribuição CompartilhaIgual 4.0 Internacional (CC BY-SA 4.0). Lista de colaboradores do site \url{https://github.com/reamat/PreCalculo/graphs/contributors}.

\noindent Baseado no template de Goro Akechi, com Licença Creative Commons Atribuição NãoComercial CompartilhaIgual 4.0 Internacional (CC BY-NC-SA 4.0). Disponível em: \url{https://www.latextemplates.com/template/legrand-orange-book} % Copyright notice

\vspace{.5cm}
%\noindent \textsc{Published by Publisher}\\ % Publisher

%\noindent Repositório: \url{https://sites.google.com/view/jordanlambert/} % URL

\vspace{.5cm}
\noindent O conteúdo deste trabalho está licenciado sob a Licença Creative Commons Atribuição CompartilhaIgual 4.0 Internacional. Para ver uma cópia desta licença, visite
\url{https://creativecommons.org/licenses/by-sa/4.0/} ou envie uma carta para Creative Commons, PO Box 1866, Mountain View, CA 94042, USA. % License information, replace this with your own license (if any)

\vspace{.5cm}
\noindent \textit{Última atualização, \today} % Printing/edition date

%----------------------------------------------------------------------------------------
%	TABLE OF CONTENTS
%----------------------------------------------------------------------------------------

\pagestyle{empty} % Disable headers and footers for the following pages

\chapter*{Prefácio}

Este texto tem como objetivo apresentar todo o conteúdo da disciplina de Introdução à Matemática Superior oferecido aos cursos de Matemática e Física do Instituto de Ciências Exatas de Volta Redonda da Universidade Federal Fluminense (ICEx-UFF). Esta disciplina foi criada em 2010, juntamente com os cursos do ICEx para fornecer a base matemática necessária aos alunos para posteriormente cursarem a disciplina de Cálculo I.

Há diversos livros de Pré-Cálculo disponíveis, no entanto percebo que a maioria deles são extremamente densos para o estudo em um período letivo, além do fato de que vários estudante chegam com diferentes dificuldades nos conteúdos de Ensino Médio, que foram agravadas durante os anos de pandemia.

Durante muitos anos, utilizamos boas apostilas elaboradas pela Profª Marina Sequeiros, mas percebi a necessidade de uma atualização deste material. Em 2023, encontrei o material de ``Pré-Cálculo'' colaborativo organizado pelos professores Francieli Triches e Helder Geovane Gomes de Lima, a qual disponibilizaram todo material em LaTeX no GitHub. Com este material como base, estou elaborando, com ajuda da professora Marina Ribeiro, este novo texto para que possa ser utilizado nos anos seguintes.

Os Capítulos foram elaborados de forma que cada um corresponda a uma aula teórica.
%Para organização, eles foram divididos em cinco partes: Aritimética Básica, Equações e Inequações, Polinômios, Funções Reais I e II. 
Os Capítulos contam com teoria, exemplos para contextualização e, eventualmente, links de vídeos e materiais de Geogebra para auxiliar no estudo e aprofundar o conhecimento dos estudantes. Ao final de cada capítulo, é apresentado um lista de exercícios que é fundamental para treinar e aprender.
%No total, são 18 Capítulos o qual apresentamos \total{totalExemplos} exemplos e propomos \total{totalExercicios} exercícios para resolução.

Ao final do texto, apresentamos outras referências bibliográficas, respostas dos exercícios (\textit{ainda em elaboração}) e um Apêndice com conteúdos adicionais que podem ser úteis ao estudante, mas não são o objetivo didático desta disciplina.

\setcounter{tocdepth}{0}
%\tableofcontents % Output the table of contents